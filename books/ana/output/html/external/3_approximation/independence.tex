\documentclass[11pt]{article}
\usepackage{fancyhdr}
\usepackage{amsmath}
\usepackage{graphicx}

\usepackage[top=1in, bottom=1in, left=1in, right=1in]{geometry}
\usepackage{color}
\newcommand{\compactlist}{\setlength{\itemsep}{0pt} \setlength{\parskip}{0pt} \setlength{\leftskip}{-1em}}

\lhead{MATH 4263/5373 ANA}
\rhead{Fall 2019}
\chead[RE]{\qquad\qquad\qquad\bf{Linear independence}}
\cfoot{}

\begin{document}
\pagestyle{fancy}

%

Consider ordered pairs \((1, 0)\) and \((0, 1)\) (or vectors \(\left({1}\atop{0}\right)\) and \(\left({0}\atop{1}\right)\) if you have some background in linear algebra).  The main idea of linear independence is that we can express other points (or vectors) in terms of these two. To get the point  \((a, b)\) (or vector \(\left({a}\atop{b}\right)\)), we would take \[a\cdot(1, 0) + b\cdot(0, 1) =  (a, b)\text{ or }  a\left({1}\atop{0}\right) + b\left({0}\atop{1}\right) = \left({a\atop b}\right)\]

Notice that if we took \((k, 0)\) instead of \((0, 1)\) as our second pair (or vector), we could not express a point with a nonzero \(y\)-coordinate in terms of these two.  Additionally, we could choose \[k\cdot(1, 0) + (-1)\cdot(k, 0) =  (0, 0)\] for arbitrary (nonzero) \(k\).  If we used our original two points (or vectors), the only way to combine these to get the point at the origin is to multiply each by zero \[0\cdot(1, 0) + 0\cdot(0, 1) =  (0, 0)\] which satisfies the definition for linear independence.
\end{document}