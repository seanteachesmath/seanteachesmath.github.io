%********************************************%
%*       Generated from PreTeXt source      *%
%*       on 2021-02-04T23:01:56-06:00       *%
%*   A recent stable commit (2020-08-09):   *%
%* 98f21740783f166a773df4dc83cab5293ab63a4a *%
%*                                          *%
%*         https://pretextbook.org          *%
%*                                          *%
%********************************************%
\documentclass[oneside,10pt,]{book}
%% Custom Preamble Entries, early (use latex.preamble.early)
%% Default LaTeX packages
%%   1.  always employed (or nearly so) for some purpose, or
%%   2.  a stylewriter may assume their presence
\usepackage{geometry}
%% Some aspects of the preamble are conditional,
%% the LaTeX engine is one such determinant
\usepackage{ifthen}
%% etoolbox has a variety of modern conveniences
\usepackage{etoolbox}
\usepackage{ifxetex,ifluatex}
%% Raster graphics inclusion
\usepackage{graphicx}
%% Color support, xcolor package
%% Always loaded, for: add/delete text, author tools
%% Here, since tcolorbox loads tikz, and tikz loads xcolor
\PassOptionsToPackage{usenames,dvipsnames,svgnames,table}{xcolor}
\usepackage{xcolor}
%% begin: defined colors, via xcolor package, for styling
%% end: defined colors, via xcolor package, for styling
%% Colored boxes, and much more, though mostly styling
%% skins library provides "enhanced" skin, employing tikzpicture
%% boxes may be configured as "breakable" or "unbreakable"
%% "raster" controls grids of boxes, aka side-by-side
\usepackage{tcolorbox}
\tcbuselibrary{skins}
\tcbuselibrary{breakable}
\tcbuselibrary{raster}
%% We load some "stock" tcolorbox styles that we use a lot
%% Placement here is provisional, there will be some color work also
%% First, black on white, no border, transparent, but no assumption about titles
\tcbset{ bwminimalstyle/.style={size=minimal, boxrule=-0.3pt, frame empty,
colback=white, colbacktitle=white, coltitle=black, opacityfill=0.0} }
%% Second, bold title, run-in to text/paragraph/heading
%% Space afterwards will be controlled by environment,
%% independent of constructions of the tcb title
%% Places \blocktitlefont onto many block titles
\tcbset{ runintitlestyle/.style={fonttitle=\blocktitlefont\upshape\bfseries, attach title to upper} }
%% Spacing prior to each exercise, anywhere
\tcbset{ exercisespacingstyle/.style={before skip={1.5ex plus 0.5ex}} }
%% Spacing prior to each block
\tcbset{ blockspacingstyle/.style={before skip={2.0ex plus 0.5ex}} }
%% xparse allows the construction of more robust commands,
%% this is a necessity for isolating styling and behavior
%% The tcolorbox library of the same name loads the base library
\tcbuselibrary{xparse}
%% Hyperref should be here, but likes to be loaded late
%%
%% Inline math delimiters, \(, \), need to be robust
%% 2016-01-31:  latexrelease.sty  supersedes  fixltx2e.sty
%% If  latexrelease.sty  exists, bugfix is in kernel
%% If not, bugfix is in  fixltx2e.sty
%% See:  https://tug.org/TUGboat/tb36-3/tb114ltnews22.pdf
%% and read "Fewer fragile commands" in distribution's  latexchanges.pdf
\IfFileExists{latexrelease.sty}{}{\usepackage{fixltx2e}}
%% Text height identically 9 inches, text width varies on point size
%% See Bringhurst 2.1.1 on measure for recommendations
%% 75 characters per line (count spaces, punctuation) is target
%% which is the upper limit of Bringhurst's recommendations
\geometry{letterpaper,total={340pt,9.0in}}
%% Custom Page Layout Adjustments (use latex.geometry)
%% This LaTeX file may be compiled with pdflatex, xelatex, or lualatex executables
%% LuaTeX is not explicitly supported, but we do accept additions from knowledgeable users
%% The conditional below provides  pdflatex  specific configuration last
%% begin: engine-specific capabilities
\ifthenelse{\boolean{xetex} \or \boolean{luatex}}{%
%% begin: xelatex and lualatex-specific default configuration
\ifxetex\usepackage{xltxtra}\fi
%% realscripts is the only part of xltxtra relevant to lualatex 
\ifluatex\usepackage{realscripts}\fi
%% end:   xelatex and lualatex-specific default configuration
}{
%% begin: pdflatex-specific default configuration
%% We assume a PreTeXt XML source file may have Unicode characters
%% and so we ask LaTeX to parse a UTF-8 encoded file
%% This may work well for accented characters in Western language,
%% but not with Greek, Asian languages, etc.
%% When this is not good enough, switch to the  xelatex  engine
%% where Unicode is better supported (encouraged, even)
\usepackage[utf8]{inputenc}
%% end: pdflatex-specific default configuration
}
%% end:   engine-specific capabilities
%%
%% Fonts.  Conditional on LaTex engine employed.
%% Default Text Font: The Latin Modern fonts are
%% "enhanced versions of the [original TeX] Computer Modern fonts."
%% We use them as the default text font for PreTeXt output.
%% Automatic Font Control
%% Portions of a document, are, or may, be affected by defined commands
%% These are perhaps more flexible when using  xelatex  rather than  pdflatex
%% The following definitions are meant to be re-defined in a style, using \renewcommand
%% They are scoped when employed (in a TeX group), and so should not be defined with an argument
\newcommand{\divisionfont}{\relax}
\newcommand{\blocktitlefont}{\relax}
\newcommand{\contentsfont}{\relax}
\newcommand{\pagefont}{\relax}
\newcommand{\tabularfont}{\relax}
\newcommand{\xreffont}{\relax}
\newcommand{\titlepagefont}{\relax}
%%
\ifthenelse{\boolean{xetex} \or \boolean{luatex}}{%
%% begin: font setup and configuration for use with xelatex
%% Generally, xelatex is necessary for non-Western fonts
%% fontspec package provides extensive control of system fonts,
%% meaning *.otf (OpenType), and apparently *.ttf (TrueType)
%% that live *outside* your TeX/MF tree, and are controlled by your *system*
%% (it is possible that a TeX distribution will place fonts in a system location)
%%
%% The fontspec package is the best vehicle for using different fonts in  xelatex
%% So we load it always, no matter what a publisher or style might want
%%
\usepackage{fontspec}
%%
%% begin: xelatex main font ("font-xelatex-main" template)
%% Latin Modern Roman is the default font for xelatex and so is loaded with a TU encoding
%% *in the format* so we can't touch it, only perhaps adjust it later
%% in one of two ways (then known by NFSS names such as "lmr")
%% (1) via NFSS with font family names such as "lmr" and "lmss"
%% (2) via fontspec with commands like \setmainfont{Latin Modern Roman}
%% The latter requires the font to be known at the system-level by its font name,
%% but will give access to OTF font features through optional arguments
%% https://tex.stackexchange.com/questions/470008/
%% where-and-how-does-fontspec-sty-specify-the-default-font-latin-modern-roman
%% http://tex.stackexchange.com/questions/115321
%% /how-to-optimize-latin-modern-font-with-xelatex
%%
%% end:   xelatex main font ("font-xelatex-main" template)
%% begin: xelatex mono font ("font-xelatex-mono" template)
%% (conditional on non-trivial uses being present in source)
%% end:   xelatex mono font ("font-xelatex-mono" template)
%% begin: xelatex font adjustments ("font-xelatex-style" template)
%% end:   xelatex font adjustments ("font-xelatex-style" template)
%%
%% Extensive support for other languages
\usepackage{polyglossia}
%% Set main/default language based on pretext/@xml:lang value
%% document language code is "en-US", US English
%% usmax variant has extra hypenation
\setmainlanguage[variant=usmax]{english}
%% Enable secondary languages based on discovery of @xml:lang values
%% Enable fonts/scripts based on discovery of @xml:lang values
%% Western languages should be ably covered by Latin Modern Roman
%% end:   font setup and configuration for use with xelatex
}{%
%% begin: font setup and configuration for use with pdflatex
%% begin: pdflatex main font ("font-pdflatex-main" template)
\usepackage{lmodern}
\usepackage[T1]{fontenc}
%% end:   pdflatex main font ("font-pdflatex-main" template)
%% begin: pdflatex mono font ("font-pdflatex-mono" template)
%% (conditional on non-trivial uses being present in source)
%% end:   pdflatex mono font ("font-pdflatex-mono" template)
%% begin: pdflatex font adjustments ("font-pdflatex-style" template)
%% end:   pdflatex font adjustments ("font-pdflatex-style" template)
%% end:   font setup and configuration for use with pdflatex
}
%% Symbols, align environment, commutative diagrams, bracket-matrix
\usepackage{amsmath}
\usepackage{amscd}
\usepackage{amssymb}
%% allow page breaks within display mathematics anywhere
%% level 4 is maximally permissive
%% this is exactly the opposite of AMSmath package philosophy
%% there are per-display, and per-equation options to control this
%% split, aligned, gathered, and alignedat are not affected
\allowdisplaybreaks[4]
%% allow more columns to a matrix
%% can make this even bigger by overriding with  latex.preamble.late  processing option
\setcounter{MaxMatrixCols}{30}
%%
%%
%% Division Titles, and Page Headers/Footers
%% titlesec package, loading "titleps" package cooperatively
%% See code comments about the necessity and purpose of "explicit" option.
%% The "newparttoc" option causes a consistent entry for parts in the ToC 
%% file, but it is only effective if there is a \titleformat for \part.
%% "pagestyles" loads the  titleps  package cooperatively.
\usepackage[explicit, newparttoc, pagestyles]{titlesec}
%% The companion titletoc package for the ToC.
\usepackage{titletoc}
%% Fixes a bug with transition from chapters to appendices in a "book"
%% See generating XSL code for more details about necessity
\newtitlemark{\chaptertitlename}
%% begin: customizations of page styles via the modal "titleps-style" template
%% Designed to use commands from the LaTeX "titleps" package
%% Plain pages should have the same font for page numbers
\renewpagestyle{plain}{%
\setfoot{}{\pagefont\thepage}{}%
}%
%% Single pages as in default LaTeX
\renewpagestyle{headings}{%
\sethead{\pagefont\slshape\MakeUppercase{\ifthechapter{\chaptertitlename\space\thechapter.\space}{}\chaptertitle}}{}{\pagefont\thepage}%
}%
\pagestyle{headings}
%% end: customizations of page styles via the modal "titleps-style" template
%%
%% Create globally-available macros to be provided for style writers
%% These are redefined for each occurence of each division
\newcommand{\divisionnameptx}{\relax}%
\newcommand{\titleptx}{\relax}%
\newcommand{\subtitleptx}{\relax}%
\newcommand{\shortitleptx}{\relax}%
\newcommand{\authorsptx}{\relax}%
\newcommand{\epigraphptx}{\relax}%
%% Create environments for possible occurences of each division
%% Environment for a PTX "chapter" at the level of a LaTeX "chapter"
\NewDocumentEnvironment{chapterptx}{mmmmmm}
{%
\renewcommand{\divisionnameptx}{Chapter}%
\renewcommand{\titleptx}{#1}%
\renewcommand{\subtitleptx}{#2}%
\renewcommand{\shortitleptx}{#3}%
\renewcommand{\authorsptx}{#4}%
\renewcommand{\epigraphptx}{#5}%
\chapter[{#3}]{#1}%
\label{#6}%
}{}%
%% Environment for a PTX "section" at the level of a LaTeX "section"
\NewDocumentEnvironment{sectionptx}{mmmmmm}
{%
\renewcommand{\divisionnameptx}{Section}%
\renewcommand{\titleptx}{#1}%
\renewcommand{\subtitleptx}{#2}%
\renewcommand{\shortitleptx}{#3}%
\renewcommand{\authorsptx}{#4}%
\renewcommand{\epigraphptx}{#5}%
\section[{#3}]{#1}%
\label{#6}%
}{}%
%% Environment for a PTX "subsection" at the level of a LaTeX "subsection"
\NewDocumentEnvironment{subsectionptx}{mmmmmm}
{%
\renewcommand{\divisionnameptx}{Subsection}%
\renewcommand{\titleptx}{#1}%
\renewcommand{\subtitleptx}{#2}%
\renewcommand{\shortitleptx}{#3}%
\renewcommand{\authorsptx}{#4}%
\renewcommand{\epigraphptx}{#5}%
\subsection[{#3}]{#1}%
\label{#6}%
}{}%
%%
%% Styles for six traditional LaTeX divisions
\titleformat{\part}[display]
{\divisionfont\Huge\bfseries\centering}{\divisionnameptx\space\thepart}{30pt}{\Huge#1}
[{\Large\centering\authorsptx}]
\titleformat{\chapter}[display]
{\divisionfont\huge\bfseries}{\divisionnameptx\space\thechapter}{20pt}{\Huge#1}
[{\Large\authorsptx}]
\titleformat{name=\chapter,numberless}[display]
{\divisionfont\huge\bfseries}{}{0pt}{#1}
[{\Large\authorsptx}]
\titlespacing*{\chapter}{0pt}{50pt}{40pt}
\titleformat{\section}[hang]
{\divisionfont\Large\bfseries}{\thesection}{1ex}{#1}
[{\large\authorsptx}]
\titleformat{name=\section,numberless}[block]
{\divisionfont\Large\bfseries}{}{0pt}{#1}
[{\large\authorsptx}]
\titlespacing*{\section}{0pt}{3.5ex plus 1ex minus .2ex}{2.3ex plus .2ex}
\titleformat{\subsection}[hang]
{\divisionfont\large\bfseries}{\thesubsection}{1ex}{#1}
[{\normalsize\authorsptx}]
\titleformat{name=\subsection,numberless}[block]
{\divisionfont\large\bfseries}{}{0pt}{#1}
[{\normalsize\authorsptx}]
\titlespacing*{\subsection}{0pt}{3.25ex plus 1ex minus .2ex}{1.5ex plus .2ex}
\titleformat{\subsubsection}[hang]
{\divisionfont\normalsize\bfseries}{\thesubsubsection}{1em}{#1}
[{\small\authorsptx}]
\titleformat{name=\subsubsection,numberless}[block]
{\divisionfont\normalsize\bfseries}{}{0pt}{#1}
[{\normalsize\authorsptx}]
\titlespacing*{\subsubsection}{0pt}{3.25ex plus 1ex minus .2ex}{1.5ex plus .2ex}
\titleformat{\paragraph}[hang]
{\divisionfont\normalsize\bfseries}{\theparagraph}{1em}{#1}
[{\small\authorsptx}]
\titleformat{name=\paragraph,numberless}[block]
{\divisionfont\normalsize\bfseries}{}{0pt}{#1}
[{\normalsize\authorsptx}]
\titlespacing*{\paragraph}{0pt}{3.25ex plus 1ex minus .2ex}{1.5em}
%%
%% Styles for five traditional LaTeX divisions
\titlecontents{part}%
[0pt]{\contentsmargin{0em}\addvspace{1pc}\contentsfont\bfseries}%
{\Large\thecontentslabel\enspace}{\Large}%
{}%
[\addvspace{.5pc}]%
\titlecontents{chapter}%
[0pt]{\contentsmargin{0em}\addvspace{1pc}\contentsfont\bfseries}%
{\large\thecontentslabel\enspace}{\large}%
{\hfill\bfseries\thecontentspage}%
[\addvspace{.5pc}]%
\dottedcontents{section}[3.8em]{\contentsfont}{2.3em}{1pc}%
\dottedcontents{subsection}[6.1em]{\contentsfont}{3.2em}{1pc}%
\dottedcontents{subsubsection}[9.3em]{\contentsfont}{4.3em}{1pc}%
%%
%% Begin: Semantic Macros
%% To preserve meaning in a LaTeX file
%%
%% \mono macro for content of "c", "cd", "tag", etc elements
%% Also used automatically in other constructions
%% Simply an alias for \texttt
%% Always defined, even if there is no need, or if a specific tt font is not loaded
\newcommand{\mono}[1]{\texttt{#1}}
%%
%% Following semantic macros are only defined here if their
%% use is required only in this specific document
%%
%% End: Semantic Macros
%% Division Numbering: Chapters, Sections, Subsections, etc
%% Division numbers may be turned off at some level ("depth")
%% A section *always* has depth 1, contrary to us counting from the document root
%% The latex default is 3.  If a larger number is present here, then
%% removing this command may make some cross-references ambiguous
%% The precursor variable $numbering-maxlevel is checked for consistency in the common XSL file
\setcounter{secnumdepth}{3}
%%
%% AMS "proof" environment is no longer used, but we leave previously
%% implemented \qedhere in place, should the LaTeX be recycled
\newcommand{\qedhere}{\relax}
%%
%% A faux tcolorbox whose only purpose is to provide common numbering
%% facilities for most blocks (possibly not projects, 2D displays)
%% Controlled by  numbering.theorems.level  processing parameter
\newtcolorbox[auto counter, number within=section]{block}{}
%%
%% This document is set to number PROJECT-LIKE on a separate numbering scheme
%% So, a faux tcolorbox whose only purpose is to provide this numbering
%% Controlled by  numbering.projects.level  processing parameter
\newtcolorbox[auto counter, number within=section]{project-distinct}{}
%% A faux tcolorbox whose only purpose is to provide common numbering
%% facilities for 2D displays which are subnumbered as part of a "sidebyside"
\makeatletter
\newtcolorbox[auto counter, number within=tcb@cnt@block, number freestyle={\noexpand\thetcb@cnt@block(\noexpand\alph{\tcbcounter})}]{subdisplay}{}
\makeatother
%% Localize LaTeX supplied names (possibly none)
\renewcommand*{\chaptername}{Chapter}
%% More flexible list management, esp. for references
%% But also for specifying labels (i.e. custom order) on nested lists
\usepackage{enumitem}
%% hyperref driver does not need to be specified, it will be detected
%% Footnote marks in tcolorbox have broken linking under
%% hyperref, so it is necessary to turn off all linking
%% It *must* be given as a package option, not with \hypersetup
\usepackage[hyperfootnotes=false]{hyperref}
%% Hyperlinking active in electronic PDFs, all links solid and blue
\hypersetup{colorlinks=true,linkcolor=blue,citecolor=blue,filecolor=blue,urlcolor=blue}
\hypersetup{pdftitle={calculus}}
%% If you manually remove hyperref, leave in this next command
\providecommand\phantomsection{}
%% Graphics Preamble Entries
\usepackage{tikz}
%% If tikz has been loaded, replace ampersand with \amp macro
%% extpfeil package for certain extensible arrows,
%% as also provided by MathJax extension of the same name
%% NB: this package loads mtools, which loads calc, which redefines
%%     \setlength, so it can be removed if it seems to be in the 
%%     way and your math does not use:
%%     
%%     \xtwoheadrightarrow, \xtwoheadleftarrow, \xmapsto, \xlongequal, \xtofrom
%%     
%%     we have had to be extra careful with variable thickness
%%     lines in tables, and so also load this package late
\usepackage{extpfeil}
%% Custom Preamble Entries, late (use latex.preamble.late)
%% Begin: Author-provided packages
%% (From  docinfo/latex-preamble/package  elements)
%% End: Author-provided packages
%% Begin: Author-provided macros
%% (From  docinfo/macros  element)
%% Plus three from MBX for XML characters
\newcommand{\foo}{bar}
    \newcommand{\compactlist}{\setlength{\itemsep}{0pt} \setlength{\parskip}{0pt} \setlength{\leftskip}{-1em}}
	\newcommand{\ds}{\displaystyle}
	\newcommand{\blank}{\underline{\hspace{2cm}}}
\newcommand{\lt}{<}
\newcommand{\gt}{>}
\newcommand{\amp}{&}
%% End: Author-provided macros
\begin{document}
%
%
\typeout{************************************************}
\typeout{Chapter 1 My First Chapter}
\typeout{************************************************}
%
\begin{chapterptx}{My First Chapter}{}{My First Chapter}{}{}{x:chapter:my-first-chapter}
%
%
\typeout{************************************************}
\typeout{Section 1.1 Limits}
\typeout{************************************************}
%
\begin{sectionptx}{Limits}{}{Limits}{}{}{g:section:idp140189372663384}
%
%
\typeout{************************************************}
\typeout{Subsection 1.1.1 Concept of a limit}
\typeout{************************************************}
%
\begin{subsectionptx}{Concept of a limit}{}{Concept of a limit}{}{}{g:subsection:idp140189372664168}
The value, if any, that function values approach (in the \(y\)-direction) as the arguments approach a particular number (in the \(x\)-direction).  If, from both sides of a particular argument, the function approaches a particular value, that value is the limit.  If, from both sides of a particular argument, the function increases (or decreases) without bound, we say the limit is \(\infty\) (or \(-\infty\)).  If the limits from both sides differ, we say the two-sided limit does not exist (DNE).%
%
\begin{itemize}[label=\textbullet]
\item{}Limits can be calculated algebraically or read from a graph.%
\item{}Points on graphs marked by open circles indicate a function is not defined by that \(y\)-value.  Points on graphs marked by closed or filled circles indicate a function is defined by that \(y\)-value.%
\item{}The two-sided limit takes its value when the values of the two one-sided limits match.  Verify that we would conclude that \(\displaystyle \lim_{x\to0}\dfrac{1}{x^2} = \infty\), but \(\displaystyle \lim_{x\to0}\dfrac{1}{x} = \text{D.N.E}\).%
\end{itemize}
\end{subsectionptx}
%
%
\typeout{************************************************}
\typeout{Subsection 1.1.2 Computing limits}
\typeout{************************************************}
%
\begin{subsectionptx}{Computing limits}{}{Computing limits}{}{}{g:subsection:idp140189372674248}
Limits involving zero\textbraceright{} If, when evaluated by substitution, a limit of a rational function is of the form:%
%
\begin{itemize}[label=\textbullet]
\item{}[form "\(\frac{0}{\#}\)":] the value of the limit is \(0\)%
\item{}[form "\(\frac{0}{0}\)":] the expression in the limit can be simplified by factoring, rationalization%
\item{}[form "\(\frac{\#}{0}\)":] the value of the limit technically does not exist and must be investigated by one-sided limits and\slash{}or a test of signs to determine if the value is \(\pm\infty\) or DNE%
\end{itemize}
\end{subsectionptx}
%
%
\typeout{************************************************}
\typeout{Subsection 1.1.3 Limits at infinity}
\typeout{************************************************}
%
\begin{subsectionptx}{Limits at infinity}{}{Limits at infinity}{}{}{g:subsection:idp140189372679800}
%
\begin{itemize}[label=\textbullet]
\item{}[end behavior:] power functions (those of the form \(f(x) = x^n\) for \(n>0\)) approach \(\infty\) as \(x\to\infty\). Those with even powers approach \(\infty\) as \(x\to-\infty\), while those with odd powers approach \(-\infty\) as \(x\to-\infty\)%
\item{}[leading terms:] for limits of rational functions, it is sufficient to take the limit of the ratio of leading terms%
\end{itemize}
\end{subsectionptx}
%
%
\typeout{************************************************}
\typeout{Subsection 1.1.4 Continuity}
\typeout{************************************************}
%
\begin{subsectionptx}{Continuity}{}{Continuity}{}{}{g:subsection:idp140189372687928}
%
\begin{itemize}[label=\textbullet]
\item{}[limits:] By definition of continuity at a point \(\displaystyle\lim_{x\to a} f(x) = f(a)\), meaning we can easily evaluate limits of continuous functions by direct substitution (i.e., function evaluation)%
\item{}[discontinuity:] Functions may have jump discontinuities (piecewise functions), holes or removable discontinuities (common factors of zero on top and bottom), infinite discontinuities (division of a number by zero)%
\end{itemize}
\end{subsectionptx}
\end{sectionptx}
%
%
\typeout{************************************************}
\typeout{Section 1.2 Derivatives (Differentiation)}
\typeout{************************************************}
%
\begin{sectionptx}{Derivatives (Differentiation)}{}{Derivatives (Differentiation)}{}{}{g:section:idp140189372691384}
%
%
\typeout{************************************************}
\typeout{Subsection 1.2.1 Rates of change}
\typeout{************************************************}
%
\begin{subsectionptx}{Rates of change}{}{Rates of change}{}{}{g:subsection:idp140189372692472}
Given a function \(f(x)\) and two points \((x, f(x))\) and \((x+h, f(x+h))\),%
\begin{itemize}[label=\textbullet]
\item{}[avg. r.o.c:] \(\displaystyle m_{sec} = \dfrac{f(x+h)-f(x)}{h}\) gives the slope of the secant line connecting these points on the graph of the function and describes the average rate of change of the function over the interval%
\item{}[inst. r.o.c:] \(\displaystyle m_{tan} = f'(x) = \dfrac{df}{dx} = \lim_{h\to0} \dfrac{f(x+h)-f(x)}{h}\) gives the slope of the tangent line touching the graph at a point \((x, f(x))\) by the \textbackslash{}textit\textbraceleft{}limit definition of the derivative\textbraceright{} and describes the instantaneous rate of change of the function at the point%
\end{itemize}
Notice that \(\displaystyle m_{tan} = \lim_{h\to0} m_{sec}\).%
\end{subsectionptx}
%
%
\typeout{************************************************}
\typeout{Subsection 1.2.2 Basic derivatives}
\typeout{************************************************}
%
\begin{subsectionptx}{Basic derivatives}{}{Basic derivatives}{}{}{g:subsection:idp140189372693736}
%
\begin{itemize}[label=\textbullet]
\item{}[power rule:] \(\dfrac{d}{dx}\left(x^n\right) = nx^{n-1}\) (multiply by the old power, then reduce the power by one)%
\item{}[polynomials:] Polynomials can be differentiated term by term by repeated application of the power rule to each term.%
\end{itemize}
\end{subsectionptx}
%
%
\typeout{************************************************}
\typeout{Subsection 1.2.3 Products and Quotients}
\typeout{************************************************}
%
\begin{subsectionptx}{Products and Quotients}{}{Products and Quotients}{}{}{g:subsection:idp140189372702648}
%
\begin{itemize}[label=\textbullet]
\item{}[product rule:] \(\displaystyle \dfrac{d}{dx}\left(F(x)S(x)\right) = F(x)S'(x) + S(x)F'(x)\)%
\item{}[quotient rule:] \(\displaystyle \dfrac{d}{dx}\left(\dfrac{T(x)}{B(x)}\right) = \dfrac{B(x)T'(x) - T(x)B'(x)}{(B(x))^2}\)%
\end{itemize}
\end{subsectionptx}
%
%
\typeout{************************************************}
\typeout{Subsection 1.2.4 Trigonometric functions (limits and derivative)}
\typeout{************************************************}
%
\begin{subsectionptx}{Trigonometric functions (limits and derivative)}{}{Trigonometric functions (limits and derivative)}{}{}{g:subsection:idp140189372704168}
The following two special limits are each of the form \(\dfrac{0}{0}\) and show up when applying the limit definition of the derivative to trigonometric functions,%
\begin{itemize}[label=\textbullet]
\item{}\(\displaystyle \displaystyle\lim_{x\to0}\dfrac{\sin(x)}{x} = 1\)%
\item{}\(\displaystyle \displaystyle\lim_{x\to0}\dfrac{1-\cos(x)}{x} = 0\)%
\end{itemize}
%
\par
Since sines and cosines are continuous, we can move the limit inside the sine or cosine, take the limit of the terms inside, and finally evaluate the sine or cosine,%
\begin{equation*}
\begin{aligned}
\displaystyle\lim_{x\to\infty}\sin\left(\dfrac{3\pi x^5+ \text{lower order terms}}{2x^5+\text{lower order terms}}\right)
\amp = \sin\left(\lim_{x\to\infty}\dfrac{3\pi x^5+\dots}{2x^5+\dots}\right)\\
\amp = \sin\left(\lim_{x\to\infty}\dfrac{3\pi x^5}{2x^5}\right)\\
\amp = \sin\left(\lim_{x\to\infty}\dfrac{3\pi}{2}\right)\\
\amp = \dots
\end{aligned}
\end{equation*}
To finish the problem take the limit then evaluate the sine.%
%
\begin{itemize}[label=\textbullet]
\item{}Derivatives of sine and cosine can be calculated by the definition.%
\item{}Derivatives of the other trigonometric functions can be calculated by the quotient rule.%
\end{itemize}
\end{subsectionptx}
%
%
\typeout{************************************************}
\typeout{Subsection 1.2.5 Chain Rule, implicit differentiation, and related rates}
\typeout{************************************************}
%
\begin{subsectionptx}{Chain Rule, implicit differentiation, and related rates}{}{Chain Rule, implicit differentiation, and related rates}{}{}{g:subsection:idp140189372711320}
For derivatives of compositions \(\displaystyle \dfrac{d}{dx}\left(f(g(x))\right) = f'(g(x))g'(x) = \dfrac{df}{dg}\cdot\dfrac{dg}{dx}\)%
\par
Implicit differentiation, an application of the chain rule, is useful when a relationship is defined by an equation too complicated (or inconvenient) to solve for \(y\).%
\begin{itemize}[label=\textbullet]
\item{}We differentiate each side of an equation and solve for \(\dfrac{dy}{dx}\).%
\item{}One application of this is to find slopes or equations of lines tangent to complicated curves at points on those curves.%
\end{itemize}
%
\par
Applications of implicit differentiation and the chain rule, related rates problems are used to relate two rates of change by an underlying formula or geometric relationship.  This is done with a 5-step process that includes a picture.%
\end{subsectionptx}
\end{sectionptx}
%
%
\typeout{************************************************}
\typeout{Section 1.3 Integrals (Integration)}
\typeout{************************************************}
%
\begin{sectionptx}{Integrals (Integration)}{}{Integrals (Integration)}{}{}{g:section:idp140189372715544}
%
%
\typeout{************************************************}
\typeout{Subsection 1.3.1 Antiderivatives and indefinite integrals}
\typeout{************************************************}
%
\begin{subsectionptx}{Antiderivatives and indefinite integrals}{}{Antiderivatives and indefinite integrals}{}{}{g:subsection:idp140189372715928}
We are interested in determining \(\displaystyle \dfrac{d}{dx}\Big(\,\,?\,\,\Big) = f(x)\).%
\begin{itemize}[label=\textbullet]
\item{}[antiderivatives:] We call the function \(F(x)\) an antiderivative, and write \(\displaystyle F(x) = \int f(x)\,dx\).%
\item{}[indefinite integral:] The collection of all possible antiderivatives is called the indefinite integral \(\displaystyle \int f(x)\,dx = F(x) + c\).%
\item{}[power rule:] For \(n\neq1\), \(\displaystyle \int x^n\,dx = \dfrac{x^{n+1}}{n+1}+c\) (add one to the power, then divide by the new power).%
\end{itemize}
%
\end{subsectionptx}
%
%
\typeout{************************************************}
\typeout{Subsection 1.3.2 \(u\)-substitution}
\typeout{************************************************}
%
\begin{subsectionptx}{\(u\)-substitution}{}{\(u\)-substitution}{}{}{g:subsection:idp140189372721032}
This classic technique is used to evaluate more complicated integrals.  We choose \(u = g(x)\) and compute the differential \(du = g'(x)dx\). We substitute these into the original integral, and translate from \(x\) and \(dx\) to \(u\) and \(du\).  In substituting,%
\begin{itemize}[label=\textbullet]
\item{}we sometimes have to multiply or divide both sides of the definition for \(du\) by a constant.  We \textbackslash{}textit\textbraceleft{}never\textbraceright{} solve completely for \(dx\) by dividing by expressions involving \(x\).%
\item{}Sometimes, for complicated problems, we are required to solve \(u = g(x)\) for \(x\) as part of the substitution step.%
\end{itemize}
%
\end{subsectionptx}
%
%
\typeout{************************************************}
\typeout{Subsection 1.3.3 Riemann sums, summation notation, and definite integrals}
\typeout{************************************************}
%
\begin{subsectionptx}{Riemann sums, summation notation, and definite integrals}{}{Riemann sums, summation notation, and definite integrals}{}{}{g:subsection:idp140189372721800}
%
\begin{itemize}[label=\textbullet]
\item{}[Riemann sums:] The area under a function (above the axis) can be approximated by collections of rectangles, this idea is used to define the definite integral.%
\item{}[algebra of sums:] In making the exact calculation, the manipulation of sums is necessary.%
%
\begin{itemize}[label=$\circ$]
\item{}\(\displaystyle \sum_{k=1}^n k = \dfrac{n(n+1)}{2}\) (\textbackslash{}textit\textbraceleft{}If the original lower index is not \(k=1\), a change of index must be performed!\textbraceright{})%
\item{}\(\displaystyle \sum_{k=1}^n c = nc\) (\textbackslash{}textit\textbraceleft{}Repeated addition gives rise to multiplication.\textbraceright{})%
\item{}\(\displaystyle \sum_{k=1}^n \left(f(x) \pm g(x)\right) = \left(\sum_{k=1}^n f(x)\right) \pm \left(\sum_{k=1}^ng(x)\right)\) (\textbackslash{}textit\textbraceleft{}The sum of a sum or difference is the sum or difference of the sums.\textbraceright{})%
\end{itemize}
\item{}[net signed area:] regions below the axis give rise to negative signed areas, the net signed area for a function over a region indicates whether there is more area above or below the axis.%
\end{itemize}
\end{subsectionptx}
%
%
\typeout{************************************************}
\typeout{Subsection 1.3.4 Definite integrals and the Fundamental Theorem of Calculus}
\typeout{************************************************}
%
\begin{subsectionptx}{Definite integrals and the Fundamental Theorem of Calculus}{}{Definite integrals and the Fundamental Theorem of Calculus}{}{}{g:subsection:idp140189372733480}
For a continuous \(f(x)\) and its antiderivative \(F(x)\), \(\displaystyle\int_a^b f(x)\,dx = F(x)\Big|_a^b = F(b) - F(a)\)%
\end{subsectionptx}
\end{sectionptx}
%
%
\typeout{************************************************}
\typeout{Section 1.4 Applications of derivatives}
\typeout{************************************************}
%
\begin{sectionptx}{Applications of derivatives}{}{Applications of derivatives}{}{}{g:section:idp140189372735144}
%
%
\typeout{************************************************}
\typeout{Subsection 1.4.1 Intervals of increasing, decreasing, and concavity}
\typeout{************************************************}
%
\begin{subsectionptx}{Intervals of increasing, decreasing, and concavity}{}{Intervals of increasing, decreasing, and concavity}{}{}{g:subsection:idp140189372735528}
%
\begin{itemize}[label=\textbullet]
\item{}[critical point:] a location where the derivative is zero or undefined%
%
\begin{itemize}[label=$\circ$]
\item{}[stationary point:] a location where the derivative is zero, a special type of critical point%
\item{}[undefined derivatives:] may occur, for example, with rational functions where the denominator is zero. To identify, set the denominator equal to zero and attempt to solve.%
\end{itemize}
\item{}[inflection point:] a location where the second derivative is zero and the function changes concavity%
\item{}[increase\slash{}decrease:] specified on closed intervals, where possible, this is indicated by the sign of the first derivative (derivative is positive, function is increasing; derivative is negative, function is decreasing)%
\item{}[concavity:] specified only on open intervals, this is indicated by the sign of the second derivative (second derivative is positive, function is concave up; second derivative is negative, function is concave down)%
\end{itemize}
\end{subsectionptx}
%
%
\typeout{************************************************}
\typeout{Subsection 1.4.2 Relative and absolute extrema}
\typeout{************************************************}
%
\begin{subsectionptx}{Relative and absolute extrema}{}{Relative and absolute extrema}{}{}{g:subsection:idp140189372739624}
%
\begin{itemize}[label=\textbullet]
\item{}Relative extrema occur at critical points and can be identified by the first (changes in function behavior) or second (identification of concavity) derivative test%
\item{}Absolute extrema can occur at critical points or at endpoints on specified intervals.%
\end{itemize}
\end{subsectionptx}
%
%
\typeout{************************************************}
\typeout{Subsection 1.4.3 Applied optimization}
\typeout{************************************************}
%
\begin{subsectionptx}{Applied optimization}{}{Applied optimization}{}{}{g:subsection:idp140189372741080}
The function to be optimized describes a cost, profit, or amount of materials. Physical or practical constraints help specify the interval for optimization.  The regular optimization procedures are followed.%
\end{subsectionptx}
\end{sectionptx}
\end{chapterptx}
\end{document}